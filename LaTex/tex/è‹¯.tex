\documentclass{ctexart}

\usepackage{chemfig}
\usepackage{mhchem} %化学式宏包
\usepackage{tcolorbox} % 用于添加边框
\usepackage{mdframed}
\usepackage{amsmath} % 支持更好的公式对齐

\title{烷烃与环烷烃的反应方程式}
\author{ZonesTissyc}
\date{\today}



\begin{document}
\section{烷烃的化学性质}


\text{酮与氰化氢加成:}\ce{\chemfig{CH_3C(=[1]O)(-[7]CH_3)} + \chemfig{CN(-[2]H)} ->T[催化剂] \chemfig{CH_3(-[0]C(-[2]OH)(-[0]CN)(-[6]CH3))}}


\text{苯:}\ce{\chemfig{[,0.7]*6(-=-=-=)}}

\text{多极碳:}\ce{\chemfig{C(-[1,0.7]OH)(-[2,0.7]OH)(-[3,0.7]OH)(-[4,0.7]OH)(-[5,0.7]OH)(-[6,0.7]OH)(-[7,0.7]OH)(-[0,1]OH)}}

\vspace{1cm}
{\CJKfamily{zhhei}1.氧化反应}
\[
\ce{2C_nH_{2n+2} + (3\textit{n} +1) O_2 -> 2n CO_2 + 2(\textit{n} +1) H_2O}
\]

\begin{tcolorbox}[colframe=blue!70, colback=blue!5, boxrule=1mm, arc=4mm, width=\textwidth, enlarge left by=0mm]
\[
\ce{2C_nH_{2n+2} + (3\textit{n} +1) O_2 -> 2n CO_2 + 2(\textit{n} +1) H_2O}
\]
\end{tcolorbox}

{\CJKfamily{zhhei}2.热裂反应}

\[
\ce{CH3CH2CH2CH3 -> *CH3 + *CH2CH2CH3}
\]

\[
\ce{CH3CH2CH2CH3 -> CH3CH2* + *CH2CH3}
\]
\end{document}
